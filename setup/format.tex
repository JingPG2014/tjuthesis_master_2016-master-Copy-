% !Mode:: "TeX:UTF-8"
%  Authors: 张井   Jing Zhang: prayever@gmail.com     天津大学2010级管理与经济学部信息管理与信息系统专业硕士生
%           余蓝涛 Lantao Yu: lantaoyu1991@gmail.com  天津大学2008级精密仪器与光电子工程学院测控技术与仪器专业本科生

%%%%%%%%%% Fonts Definition and Basics %%%%%%%%%%%%%%%%%
\newcommand{\song}{\CJKfamily{song}}    % 宋体
\newcommand{\fs}{\CJKfamily{fs}}        % 仿宋体
\newcommand{\kai}{\CJKfamily{kai}}      % 楷体
\newcommand{\hei}{\CJKfamily{hei}}      % 黑体
\newcommand{\li}{\CJKfamily{li}}        % 隶书
\newcommand{\yihao}{\fontsize{26pt}{26pt}\selectfont}       % 一号, 1.倍行距
\newcommand{\xiaoyi}{\fontsize{24pt}{24pt}\selectfont}      % 小一, 1.倍行距
\newcommand{\erhao}{\fontsize{22pt}{1.25\baselineskip}\selectfont}       % 二号, 1.25倍行距
\newcommand{\xiaoer}{\fontsize{18pt}{18pt}\selectfont}      % 小二, 单倍行距
\newcommand{\sanhao}{\fontsize{16pt}{16pt}\selectfont}      % 三号, 1.倍行距
\newcommand{\xiaosan}{\fontsize{15pt}{15pt}\selectfont}     % 小三, 1.倍行距
\newcommand{\sihao}{\fontsize{14pt}{14pt}\selectfont}       % 四号, 1.0倍行距
\newcommand{\xiaosi}{\fontsize{12pt}{20pt}\selectfont}      % 小四, 1.倍行距
\newcommand{\wuhao}{\fontsize{10.5pt}{10.5pt}\selectfont}   % 五号, 单倍行距
\newcommand{\xiaowu}{\fontsize{9pt}{9pt}\selectfont}        % 小五, 单倍行距

%\CJKcaption{gb_452}
\CJKtilde  % 重新定义了波浪符~的意义
\newcommand\prechaptername{第}
\newcommand\postchaptername{章}

% 调整罗列环境的布局
\setitemize{leftmargin=3em,itemsep=0em,partopsep=0em,parsep=0em,topsep=-0em}
\setenumerate{leftmargin=3em,itemsep=0em,partopsep=0em,parsep=0em,topsep=0em}
%\setlength{\baselineskip}{20pt}
%\renewcommand{\baselinestretch}{1.38} % 设置行距

%避免宏包 hyperref 和 arydshln 不兼容带来的目录链接失效的问题。
\def\temp{\relax}
\let\temp\addcontentsline
\gdef\addcontentsline{\phantomsection\temp}

% 自定义项目列表标签及格式 \begin{publist} 列表项 \end{publist}
\newcounter{pubctr} %自定义新计数器
\newenvironment{publist}{%%%%%定义新环境
\begin{list}{[\arabic{pubctr}]} %%标签格式
    {
     \usecounter{pubctr}
     \setlength{\leftmargin}{2.5em}     % 左边界 \leftmargin =\itemindent + \labelwidth + \labelsep
     \setlength{\itemindent}{0em}     % 标号缩进量
     \setlength{\labelsep}{1em}       % 标号和列表项之间的距离,默认0.5em
     \setlength{\rightmargin}{0em}    % 右边界
     \setlength{\topsep}{0ex}         % 列表到上下文的垂直距离
     \setlength{\parsep}{0ex}         % 段落间距
     \setlength{\itemsep}{0ex}        % 标签间距
     \setlength{\listparindent}{0pt} % 段落缩进量
    }}
{\end{list}}%%%%%

\makeatletter
\renewcommand\normalsize{
  \@setfontsize\normalsize{12pt}{12pt} % 小四对应12pt
  \setlength\abovedisplayskip{4pt}
  \setlength\abovedisplayshortskip{4pt}
  \setlength\belowdisplayskip{\abovedisplayskip}
  \setlength\belowdisplayshortskip{\abovedisplayshortskip}
  \setlength{\baselineskip}{20pt} % 设置固定行间距为20pt
  \let\@listi\@listI}
\def\defaultfont{\renewcommand{\baselinestretch}{1.0}\normalsize\selectfont}
% 设置行距和段落间垂直距离
\renewcommand{\CJKglue}{\hskip -0.08pt plus 0.08\baselineskip} % 每行大概35个字符

\makeatother
%%%%%%%%%%%%% Contents %%%%%%%%%%%%%%%%% 目录样式修改,(天津大学关于博士、硕士学位论文统一格式的规定 2016.10.24)

\renewcommand{\contentsname}{目\qquad录}
\setcounter{tocdepth}{2}
\titlecontents{chapter}[0em]{\vspace{0pt}\xiaosi\song\bfseries}%
             {\prechaptername\thecontentslabel\postchaptername\quad}{} %
             {\hspace{.5em}\titlerule*[7pt]{.}\xiaosi\contentspage}
\titlecontents{section}[2em]{\vspace{0pt}\xiaosi\song} %
            {\thecontentslabel\quad}{} %
            {\hspace{.5em}\titlerule*[7pt]{.}\xiaosi\contentspage}
\titlecontents{subsection}[4em]{\vspace{0pt}\xiaosi\song} %
            {\thecontentslabel\quad}{} %
            {\hspace{.5em}\titlerule*[7pt]{.}\xiaosi\contentspage}
%\titlecontents{subsubsection}[6em]{\vspace{0pt}\xiaosi\song} %
%            {\thecontentslabel\quad}{} %
%            {\hspace{.5em}\titlerule*[7pt]{.}\xiaosi\contentspage}

%%%%%%%%%% Chapter and Section %%%%%%%%%%%%%%%%% 各级标题样式((天津大学关于博士、硕士学位论文统一格式的规定 2016.10.24))
\setcounter{secnumdepth}{4}
\setlength{\parindent}{2em}
\renewcommand{\chaptername}{\prechaptername\thechapter\postchaptername}
\titleformat{\chapter}{\centering\xiaosan\hei}{\chaptername}{1em}{}
\titlespacing{\chapter}{0pt}{33pt}{33pt}
\titleformat{\section}{\sihao\hei}{\thesection}{1em}{}
\titlespacing{\section}{0pt}{21pt}{21pt}
\titleformat{\subsection}{\sihao\hei}{\thesubsection}{1em}{}
\titlespacing{\subsection}{0pt}{13pt}{13pt}
\titleformat{\subsubsection}{\xiaosi\hei}{\thesubsubsection}{1em}{}
\titlespacing{\subsubsection}{0pt}{10pt}{10pt}

%%%%%%%%%% Table, Figure and Equation %%%%%%%%%%%%%%%%%
\renewcommand{\tablename}{表} % 插表题头
\renewcommand{\figurename}{图} % 插图题头
\renewcommand{\thefigure}{\arabic{chapter}-\arabic{figure}} % 使图编号为 7-1 的格式 %\protect{~}
\renewcommand{\thetable}{\arabic{chapter}-\arabic{table}}%使表编号为 7-1 的格式
\renewcommand{\theequation}{\arabic{chapter}-\arabic{equation}}%使公式编号为 7-1 的格式
\renewcommand{\thesubfigure}{(\alph{subfigure})}%使子图编号为 (a)的格式
\renewcommand{\thesubtable}{(\alph{subtable})} %使子表编号为 (a)的格式
\makeatletter
\renewcommand{\p@subfigure}{\thefigure~} %使子图引用为 7-1 a) 的格式,母图编号和子图编号之间用~加一个空格
\makeatother


%% 定制浮动图形和表格标题样式
\makeatletter
\long\def\@makecaption#1#2{%
   \vskip\abovecaptionskip
   \sbox\@tempboxa{\centering\wuhao\song{#1\qquad #2} }%
   \ifdim \wd\@tempboxa >\hsize
     \centering\wuhao\song{#1\qquad #2} \par
   \else
     \global \@minipagefalse
     \hb@xt@\hsize{\hfil\box\@tempboxa\hfil}%
   \fi
   \vskip\belowcaptionskip}
\makeatother
\captiondelim{~~~~} %用来控制longtable表头分隔符

%%%%%%%%%% Theorem Environment %%%%%%%%%%%%%%%%%
\theoremstyle{plain}
\theorembodyfont{\song\rmfamily}
\theoremheaderfont{\hei\rmfamily}
\newtheorem{theorem}{定理~}[chapter]
\newtheorem{lemma}{引理~}[chapter]
\newtheorem{axiom}{公理~}[chapter]
\newtheorem{proposition}{命题~}[chapter]
\newtheorem{corollary}{推论~}[chapter]
\newtheorem{definition}{定义~}[chapter]
\newtheorem{conjecture}{猜想~}[chapter]
\newtheorem{example}{例~}[chapter]
\newtheorem{remark}{注~}[chapter]
\newtheorem{algorithm}{算法~}[chapter]
\newenvironment{proof}{\noindent{\hei 证明:}}{\hfill $ \square $ \vskip 4mm}
\theoremsymbol{$\square$}

%%%%%%%%%% Page: number, header and footer  %%%%%%%%%%%%%%%%%

%\frontmatter 或 \pagenumbering{roman}
%\mainmatter 或 \pagenumbering{arabic}
\makeatletter
\renewcommand\frontmatter{\clearpage
  \@mainmatterfalse
  \pagenumbering{Roman}} % 正文前罗马字体编号
\makeatother


%%%%%%%%%% References %%%%%%%%%%%%%%%%%
\renewcommand{\bibname}{参考文献}
% 重定义参考文献样式,来自thu
\makeatletter
\renewenvironment{thebibliography}[1]{%
   \chapter*{\bibname}%
   \xiaosi
   \list{\@biblabel{\@arabic\c@enumiv}}%
        {\renewcommand{\makelabel}[1]{##1\hfill}
         \setlength{\baselineskip}{17pt}
         \settowidth\labelwidth{0.5cm}
         \setlength{\labelsep}{0pt}
         \setlength{\itemindent}{0pt}
         \setlength{\leftmargin}{\labelwidth+\labelsep}
         \addtolength{\itemsep}{-0.7em}
         \usecounter{enumiv}%
         \let\p@enumiv\@empty
         \renewcommand\theenumiv{\@arabic\c@enumiv}}%
    \sloppy\frenchspacing
    \clubpenalty4000%
    \@clubpenalty \clubpenalty
    \widowpenalty4000%
    \interlinepenalty4000%
    \sfcode`\.\@m}
   {\def\@noitemerr
     {\@latex@warning{Empty `thebibliography' environment}}%
    \endlist\frenchspacing}
\makeatother

\addtolength{\bibsep}{3pt} % 增加参考文献间的垂直间距
\setlength{\bibhang}{2em} %每个条目自第二行起缩进的距离

% 参考文献引用作为上标出现
%\newcommand{\citeup}[1]{\textsuperscript{\cite{#1}}}
\makeatletter
    \def\@cite#1#2{\textsuperscript{[{#1\if@tempswa , #2\fi}]}}
\makeatother
%% 引用格式
\bibpunct{[}{]}{,}{s}{}{,}

%%%%%%%%%% Cover %%%%%%%%%%%%%%%%%
% 封面、摘要、版权、致谢格式定义
\makeatletter
\def\ctitle#1{\def\@ctitle{#1}}\def\@ctitle{}
\def\etitle#1{\def\@etitle{#1}}\def\@etitle{}
\def\caffil#1{\def\@caffil{#1}}\def\@caffil{}
\def\cmacrosubject#1{\def\@cmacrosubject{#1}}\def\@cmacrosubject{}
\def\cmacrosubjecttitle#1{\def\@cmacrosubjecttitle{#1}}\def\@cmacrosubjecttitle{}
\def\csubject#1{\def\@csubject{#1}}\def\@csubject{}
\def\csubjecttitle#1{\def\@csubjecttitle{#1}}\def\@csubjecttitle{}
\def\cgrade#1{\def\@cgrade{#1}}\def\@cgrade{}
\def\cauthor#1{\def\@cauthor{#1}}\def\@cauthor{}
\def\cauthortitle#1{\def\@cauthortitle{#1}}\def\@cauthortitle{}
\def\csupervisor#1{\def\@csupervisor{#1}}\def\@csupervisor{}
\def\csupervisortitle#1{\def\@csupervisortitle{#1}}\def\@csupervisortitle{}
\def\ccorsupervisor#1{\def\@ccorsupervisor{#1}}\def\@ccorsupervisor{}
\def\ccorsupervisortitle#1{\def\@ccorsupervisortitle{#1}}\def\@ccorsupervisortitle{}
\def\cdate#1{\def\@cdate{#1}}\def\@cdate{}
\def\declaretitle#1{\def\@declaretitle{#1}}\def\@declaretitle{}
\def\declarecontent#1{\def\@declarecontent{#1}}\def\@declarecontent{}
\def\authorizationtitle#1{\def\@authorizationtitle{#1}}\def\@authorizationtitle{}
\def\authorizationcontent#1{\def\@authorizationcontent{#1}}\def\@authorizationconent{}
\def\authorizationadd#1{\def\@authorizationadd{#1}}\def\@authorizationadd{}
\def\authorsigncap#1{\def\@authorsigncap{#1}}\def\@authorsigncap{}
\def\supervisorsigncap#1{\def\@supervisorsigncap{#1}}\def\@supervisorsigncap{}
\def\signdatecap#1{\def\@signdatecap{#1}}\def\@signdatecap{}
\long\def\cabstract#1{\long\def\@cabstract{#1}}\long\def\@cabstract{}
\long\def\eabstract#1{\long\def\@eabstract{#1}}\long\def\@eabstract{}
\def\ckeywords#1{\def\@ckeywords{#1}}\def\@ckeywords{}
\def\ekeywords#1{\def\@ekeywords{#1}}\def\@ekeywords{}

%在book文件类别下,\leftmark自动存录各章之章名,\rightmark记录节标题
\pagestyle{fancy}
%去掉章节标题中的数字 务必放到\pagestyle{fancy}之后才会起作用
%%不要注销这一行,否则页眉会变成:“第1章1  绪论”样式
\renewcommand{\chaptermark}[1]{\markboth{\chaptername~\ #1}{}}
  \fancyhf{}
  \fancyhead[C]{\song\wuhao \leftmark} % 页眉显示章节名称
  %\fancyhead[CO]{\song\wuhao \@cheading}
  %\fancyhead[CE]{\song\wuhao \@cheading}
  \fancyfoot[C]{\song\xiaowu ~\thepage~}
  \renewcommand{\headrulewidth}{0.7pt}%
  \renewcommand{\footrulewidth}{0pt}%

\fancypagestyle{plain}{% 设置开章页页眉页脚风格
    \fancyhf{}%
    \fancyhead[C]{\song\wuhao \leftmark}
    \fancyfoot[C]{\song\xiaowu ~\thepage~ } %%首页页脚格式
    \renewcommand{\headrulewidth}{0.7pt}%
    \renewcommand{\footrulewidth}{0pt}%
}

\fancypagestyle{only_foot}{% 设置摘要、目录的页眉页脚风格:无页眉((天津大学关于博士、硕士学位论文统一格式的规定 2016.10.24)貌似要这种only_foot的样式)
    \fancyhf{}%
    \fancyfoot[C]{\song\xiaowu ~\thepage~ }
    \renewcommand{\headrulewidth}{0pt}%
    \renewcommand{\footrulewidth}{0pt}%
}


\newlength{\@title@width}
\def\@put@covertitle#1{\makebox[\@title@width][s]{#1}}
% 定义封面
\def\makecover{
\clearpage{\pagestyle{empty}\cleardoublepage}
   \phantomsection
    \pdfbookmark[-1]{\@ctitle}{ctitle}

    \begin{titlepage}
      \vspace*{0.8cm}
      \begin{center}

      \vspace*{1cm}
      \begin{center}
      \renewcommand{\baselinestretch}{1.25} % 设置行距
      \song\erhao\textbf{\@ctitle} % 修改成宋体加粗 (天津大学关于博士、硕士学位论文统一格式的规定, 2016.10.24)
      \renewcommand{\baselinestretch}{1.25} % 设置行距
      \end{center}
      \vspace*{1cm}

      \vspace*{1cm}

      \begin{center}
      \renewcommand{\baselinestretch}{1.25} % 设置行距
      \song\erhao\textbf{\@etitle}  % 修改成宋体加粗1.25倍行距(英文默认变成了Times new Roman) (天津大学关于博士、硕士学位论文统一格式的规定, 2016.10.24)
      \renewcommand{\baselinestretch}{1.25} % 设置行距
      \end{center}

      \vspace*{3cm}
      \setlength{\@title@width}{5em}
      {\song\sihao
      \begin{tabular}{p{\@title@width}@{:}l}
        \@put@covertitle{\@csubjecttitle} & \@csubject \\
        \@put@covertitle{\@cauthortitle} & \@cauthor \\
        \@put@covertitle{\@csupervisortitle} & \@csupervisor \\
        \@put@covertitle{\@ccorsupervisortitle} & \@ccorsupervisor \\
      \end{tabular}
      }

  \vspace*{5cm}
  \song\sihao\@caffil \\
  \song\sihao\@cdate

\end{center}
%  另起一页: 独创性声明和学位论文版权使用授权书
\newpage
    \clearpage{\pagestyle{empty}\cleardoublepage} %去除空白页的页眉页脚(由于每章从奇数页开始因此有空白页)
    \thispagestyle{empty} %去掉页眉页脚
    \vspace*{1cm}
    \renewcommand{\baselinestretch}{1} % 设置行距
    \begin{center}\song\xiaoer{\@declaretitle}\end{center}\par
    \vspace*{0.5cm}
    \song\xiaosi{\@declarecontent}\par
    \vspace*{1cm}
    {\song\xiaosi
    \@authorsigncap \makebox[2.5cm][s]{}
    \@signdatecap \makebox[2cm][s]{} 年 \makebox[1cm][s]{} 月 \makebox[1cm][s]{} 日
    }

    \vspace*{3cm}
    \begin{center}\song\xiaoer{\@authorizationtitle}\end{center}\par
    \vspace*{1cm}
    {
    \song\xiaosi{\@authorizationcontent}

    \@authorizationadd\par
    }

    \vspace*{2cm}
    {\song\xiaosi\setlength{\parindent}{-0.45em}
    \begin{tabularx}{\textwidth}{ll}
        \@authorsigncap \makebox[3.5cm][s]{}  & \@supervisorsigncap \makebox[3.5cm][s]{}   \\
         &  \\
        \@signdatecap \makebox[1.5cm][s]{} 年 \makebox[1cm][s]{} 月 \makebox[1cm][s]{} 日 &
         \@signdatecap \makebox[1.5cm][s]{} 年 \makebox[1cm][s]{} 月 \makebox[1cm][s]{} 日 \\
    \end{tabularx}
    }
\end{titlepage}

%%%%%%%%%%%%%%%%%%%   Abstract and Keywords  %%%%%%%%%%%%%%%%%%%%%%%
\clearpage{\pagestyle{empty}\cleardoublepage} %去除空白页的页眉页脚(由于每章从奇数页开始因此有空白页)
\markboth{摘~要}{摘~要}
\addcontentsline{toc}{chapter}{摘\qquad 要}
\chapter*{\centering\song\erhao\textbf{摘\qquad 要}}
\thispagestyle{only_foot}

\setcounter{page}{1}
\song\defaultfont
\@cabstract
%\vspace{\baselineskip}

%\hangafter=1\hangindent=52.3pt\noindent   %如果取消该行注释,关键词换行时将会自动缩进
\noindent
{\hei\sihao 关键词:} \@ckeywords

%%%%%%%%%%%%%%%%%%%   English Abstract  %%%%%%%%%%%%%%%%%%%%%%%%%%%%%%
\clearpage{\pagestyle{empty}\cleardoublepage} %去除空白页的页眉页脚(由于每章从奇数页开始因此有空白页)
\markboth{ABSTRACT}{ABSTRACT}
\addcontentsline{toc}{chapter}{ABSTRACT}
\chapter*{\centering\erhao\textbf{ABSTRACT}}
\thispagestyle{only_foot}
%\vspace{\baselineskip}
\@eabstract
%\vspace{\baselineskip}

%\hangafter=1\hangindent=60pt\noindent  %如果取消该行注释,KEY WORDS换行时将会自动缩进
\noindent
{\sihao\textbf{KEY WORDS:}}  \@ekeywords
}

\makeatother
